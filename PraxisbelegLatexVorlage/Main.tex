%~~~~~~~~~~~~~~~~~~~~~~~~~~~~~~~~~~~~~~~~~~~~~~~~~~~~~~~~~~~~~~~~~~~~~~~~~~~~~~~~~~~~~~~~~~~~~~~~~~~~~~~~~~~~~~~~~~~~~~~~~~
%	Dokument: 	Praxisbeleg
%	Titel:		
%	Verfasser:	
%	Zeitraum:	X. Praxispahse
%~~~~~~~~~~~~~~~~~~~~~~~~~~~~~~~~~~~~~~~~~~~~~~~~~~~~~~~~~~~~~~~~~~~~~~~~~~~~~~~~~~~~~~~~~~~~~~~~~~~~~~~~~~~~~~~~~~~~~~~~~~

\documentclass[
	12pt,
	a4paper,
	oneside,
								% scrreprt ab hier
	headsepline,				% Linie unter der Kopfzeile	
	footnosepline,
	normalheadings,				% Überschriften normal setzen
	numbers=noenddot,			% Keinen Punkt hinter die letzte Zahl eines Kapitels 
	appendixprefix,				% Anhang
	openany,
	cleardoubleplain,
	abstracton,
	idxtotoc,					% Index soll im Inhaltsverzeichnis auftauchen
	liststotoc,
	DIV15,
	bibtotoc,
	BCOR7mm						% Bindungskorrektur: BCOR <Breite des 				Bindungsverlustes> 
]{scrreprt}
%----------------------------------------------------------------------
%	PAKETE
%----------------------------------------------------------------------
	
	\usepackage[utf8]{inputenc}
	\usepackage[hyphens]{url}	% Zeilenumbruch in URL's
	\usepackage{cite} 			% Literaturverzeichnis
	\usepackage{setspace}		% Zeilenabstände festlegen
	\usepackage[T1]{fontenc}	% EC-Schriften verwenden (vs. DC) da 8-Bit
	\usepackage{ae,aecompl}		% virtuelle-CM-Fonts
	\usepackage{booktabs}
	\usepackage{chngcntr}
	\usepackage{microtype}
	\usepackage[bottom, hang]{footmisc} %Fußnoten manipulieren
	\setlength{\footnotemargin}{0pt} %Fußnoten werden nicht mehr ab zewiter Zeile eingerückt
	\usepackage
	{
		acronym,				% Abkürzungsverzeichnis
		ngerman,				% Deutsche Trennungen (NEUE Rechtschreibung), 
		bibgerm,				% Deutsche Bibliographie (ALTE Rechtschreibung)
		calc,					% Erweiterung der arithmetischen Funktionen in LaTeX
								% wird verwendet um Titelseite zu zentrieren
		color,					% im Laufenden Text einfach mit \color{Farbe) zwischen den 
								% Farben umschalten, wobei Farbe einfach 
								% durch z.B. red, blue, black etc. ersetzt wird
								% \textcolor{farbe){Text)
		colortbl,				% Farben innerhalb Tabelle
		fancybox,				% shadowbox, doublebox, ovalbox, Ovalbox 
		fancyvrb,				% verbatim Erweiterung:
		float,					% Positionierung von Gleitobjekten
								% 'figure'- oder 'table'-Umgebung die 
								% Positionierung [H] gesetzt werden
		floatflt,				% Bilder im Fließtext
		wrapfig,
		graphicx,				
		mdwlist,				% compact list: itemize* ..
		scrdate,				% \todaysname 
		scrtime,				% \thistime
		scrpage2,				% Kopf- und Fuzeilen flexibel gestalten
		tabularx				% Blocksatzspalten
	}
	\usepackage[				%Referenzen
		colorlinks=false,
		linktocpage=true,
		hypertexnames=true,
		pdfpagelabels=true,
		pdfborder={0 0 0},
		linkcolor=red,
	    citecolor=cyan,
	    urlcolor=blue,
	    bookmarksnumbered=true
		]{hyperref}
%----------------------------------------------------------------------
%	Einstellungen
%----------------------------------------------------------------------
\pagestyle{scrheadings}					% Standart  Kopf- und Fuzeile
\cfoot[]{}								% Seitenzahlen rechts anordnen
\ofoot[\pagemark]{\pagemark}
\counterwithout{footnote}{chapter}		%Zähler wird für Fussnote nicht nach neuem Kapitel zurückgesetzt
%\setkomafont{pageheadfoot}{\small\scshape}
 % --------------------------------------------------------------------
 %Vermeiden einzelner Zeilen am Ende einer Seite oder oben auf einer neuen Seite
 \clubpenalty10000
 \widowpenalty10000
 \usepackage{pdfpages}
 \usepackage{subfigure} 
 \usepackage{graphicx}
 \usepackage{rotating}
 \usepackage{tocstyle} 			% KOMA-Script-Paket mit eigener Anleitung tocstyle.pdf.
 \usetocstyle{allwithdot}
 \setlength{\parindent}{0pt}	% Erstzeileneinzug auf 0pt	
 \topmargin = 0mm
 \headsep = 8pt
 \voffset = 0pt
 %---------------------------------------------------------------------
 % neue Umgebung für verwendete Sätze und Beispiele
 \newtheorem{bsp}{Beispiel}[chapter]
 \newtheorem{satz}{Satz}[chapter]
 
 %---------------------------------------------------------------------
 %	includeonly
 %---------------------------------------------------------------------
 \includeonly{		% Gibt an, welche Dateien der include-Befehl 
 					% tatschlich einfuegen darf.
   Metadaten,		% Variablen setzen
   Titelseite,		% Titelseite, Zusammenfassung und Inhaltsverzeichnis
   00Verzeichnisse,
   % Alle per include einzulesenden Dateien müssen hier angegeben sein!
   01ErsteKapitel,
   Anhang,
   Eidesstattliche_Erklaerung
   }
	%~~~~~~~~~~~~~~~~~~~~~~~~~~~~~~~~~~~~~~~~~~~~~~~~~~~~~~~~~~~~~~~~~~~~~
	%	Begin Textteil - Befehle nur noch lokal Wirksam
	%~~~~~~~~~~~~~~~~~~~~~~~~~~~~~~~~~~~~~~~~~~~~~~~~~~~~~~~~~~~~~~~~~~~~~

\begin{document}
	%======================================================================
%	Metadaten
%======================================================================

\newcommand{\dcsubject}{Preaxisbeleg}
\newcommand{\dctitle}{Thema}
\newcommand{\dcauthorlastname}{Nachname}
\newcommand{\dcauthorfirstname}{Vorname}
\newcommand{\dcdate}{\today}
\newcommand{\dcauthoremail}{E-Mail}
\newcommand{\dckeywords}{keyword}
\newcommand{\dcstreet}{Starße}
\newcommand{\dcplace}{PLZ, Ort}
\newcommand{\dcuni}{Uni}
\newcommand{\dccompany}{Firma}
\newcommand{\dccompanyadr}{Firma Straße}
\newcommand{\refereecompany}{Betreuer Firma}
\newcommand{\refereeuni}{Betreuer Uni}
\newcommand{\Abstand}{\\[0.4cm]}
\onehalfspacing

\renewcommand \thechapter {\arabic{chapter}}
\renewcommand \thesection {\arabic{chapter}.\arabic{section}}
\renewcommand \thesubsection {\arabic{chapter}.\arabic{section}.\arabic{subsection}}
\renewcommand \thesubsubsection {\arabic{chapter}.\arabic{section}.\arabic{subsection}.\alph{subsubsection})}
\renewcommand \theparagraph {} 

%%======================================================================
% Einstellungen des Hyperref-Paketes
\hypersetup
{
	pdftitle	= {\dctitle},
	pdfsubject	= {\dcsubject, \dcdate},
	pdfauthor	= {\dcauthorfirstname~\dcauthorlastname, \dcauthoremail},
	pdfkeywords	= {\dckeywords},
	pdfcreator	= {pdfTeX with Hyperref and Thumbpdf},
	pdfproducer	= {LaTeX, hyperref, thumbpdf}
}
%%====================================================================== 
	%======================================================================
%	Titelseite 
%======================================================================
\begin{titlepage}
	\begin{bfseries}
		\begin{center}
			%Oberer Teil des Titelblattes
			\Huge{Praxisbeleg}\\[1cm]
			\Large{\dctitle}\\[2cm]
		\end{center}
	\end{bfseries}
			\begin{tabular}{p{4,3cm}l}							%Tabulatoreinstellung
				\textbf{Vorgelegt am:} & \dcdate\\[0.8cm]
				\textbf{Von:} & \textbf{\dcauthorfirstname~\dcauthorlastname}\\  
				& \dcstreet\\
				& \dcplace\\[0.8cm]
				
				\textbf{Studiengang:} & Technische Informatik\\[0.8cm]
				
				\textbf{Studienrichtung:} & Prozessinformatik\\[0.8cm]
				
				\textbf{Seminargruppe:} & TI-XX\\[0.8cm]
				
				\textbf{Matrikelnummer:} & 0000000\\[0.8cm]
				
				\textbf{Praxispartner:} & \dccompany\\
				& \dccompanyadr\\
				& 09112 Chemnitz\\[0.4cm]
				
				\textbf{Gutachter:} & \refereecompany \\
				& \dccompany \\\\
				& \refereeuni\\
				& \dcuni\\
			\end{tabular}
\end{titlepage}


	%======================================================================
%	Themenblatt
%======================================================================
\cleardoublepage
\includepdf[pages={1}]{Bilder/Themenvorschlag.pdf}

%======================================================================
%	Inahltsverzeichnis
%======================================================================
\cleardoubleemptypage
\pagenumbering{Roman}
%\pdfbookmark{Inhaltsverzeichnis}{Inhaltsverzeichnis}
\tableofcontents

\cleardoublepage
\markboth{Themenblatt}{Themenblatt}	

%======================================================================
%	Abbildungsverzeichnis
%======================================================================
\cleardoublepage
\markboth{Abbildungsverzeichnis}{Abbildungsverzeichnis}
\listoffigures

%======================================================================
%	Abbildungsverzeichnis
%======================================================================
\cleardoublepage
\markboth{Tabllenverzeichnis}{Tabellenverzeichnis}
\listoftables

%======================================================================
%	Abkürzungsverzeichnis
%======================================================================
\cleardoublepage 
\addcontentsline{toc}{chapter}{Abkürzungsverzeichnis}
\markboth{Abkürzungsverzeichnis}{Abkürzungsverzeichnis}
\chapter*{Abkürzungsverzeichnis}
\begin{tabular}{p{2,5cm} p{12,1cm}}
	\toprule
	\textbf{Abkürzung} & \textbf{Bedeutung}\\
	\midrule
	\midrule
	%TODo hier Abkürzungen einfügen "\textbf{CBA} &  Component based Automation\\"
	\bottomrule
\end{tabular}

%%======================================================================
%%      Ende
%%======================================================================
\cleardoublepage
\pagenumbering{arabic}
	\sffamily
\markboth{ErsteSeite}{ErsteSeite}
\chapter{ErsteSeite}\label{cha:Einführung}




	

	%======================================================================
	%	Literaturverzeichnis
	%======================================================================
	
	\manualmark
	\markboth{titel}{Literaturverzeichnis}
	\bibliographystyle{alphabwl7b}
	\bibliography{literatur}
	
	%======================================================================
	%	Anhang
	%======================================================================
	
	\manualmark
	\cleardoublepage
	%======================================================================	
%	Anhang
%======================================================================
\cleardoublepage 
\addcontentsline{toc}{chapter}{Anhang}
\markboth{Anhang}{Anhang}
\chapter*{Anhang}
CD mit folgenden Inhalten:
\begin{itemize}
	\item dieses Dokument
	\item Latex Dateien
\end{itemize}
	
	%======================================================================
	%	Eidesstattliche Erklärung
	%======================================================================
	
	\manualmark
	\cleardoublepage
	\include{Eidesstattliche_Erklaerung}
	
\end{document}